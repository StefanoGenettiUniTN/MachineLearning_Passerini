\documentclass[oneside]{book}
\usepackage[english]{babel}
\usepackage{listings}
\usepackage{hyperref}
\usepackage[font=small,labelfont=bf]{caption}
\usepackage{multicol}
\usepackage[ruled,vlined,algo2e]{algorithm2e}
\usepackage[toc, page]{appendix}
\usepackage{float}
\usepackage{amsfonts} 
\usepackage{algorithm}
\usepackage{algorithmic}
% TRY USE THESE TWO INSTEAD:
% \usepackage{algpseudocode}
% \usepackage{algorithmicx}
\usepackage{amssymb}
\usepackage{derivative}
\usepackage[framemethod=tikz]{mdframed}
\usepackage{amsmath}
\usepackage{graphicx} %Loading the package
\usepackage{amsmath} %More flavours for equations
\usepackage{mathtools} %More flavours for equations
\usepackage[makeroom]{cancel} %Semplification bar
\usepackage{subcaption}
\usepackage{enumitem}
\usepackage{bbold} %identity matrix
\usepackage{multirow}

\graphicspath{images} %Setting the graphicspath
\setcounter{secnumdepth}{5}

\lstset{
    frame=tb, % draw a frame at the top and bottom of the code block
    tabsize=4, % tab space width
    showstringspaces=false, % don't mark spaces in strings
    numbers=none, % display line numbers on the left
    commentstyle=\color{green}, % comment color
    keywordstyle=\color{red}, % keyword color
    stringstyle=\color{blue}, % string color
    breaklines=true,
    postbreak=\mbox{\textcolor{green}{$\hookrightarrow$}\space}
}



\newcommand\defi[1]{\begin{mdframed}[nobreak=true,hidealllines=false,linecolor=blue!40,linewidth=2pt,backgroundcolor=blue!5,]{#1}\end{mdframed}}

\newcommand\defib[2]{\begin{mdframed}[hidealllines=false,linecolor=blue!40,linewidth=2pt,backgroundcolor=blue!5,]\begin{center}\textbf{{#1}}\end{center}\vspace{2mm}{#2}\end{mdframed}}

\newcommand\theo[1]{\begin{mdframed}[hidealllines=false,linecolor=yellow!40,linewidth=2pt,backgroundcolor=yellow!5,]{#1}\end{mdframed}}


\newcommand\sumup[1]{\begin{mdframed}[hidealllines=false,linecolor=red!40,linewidth=2pt,backgroundcolor=red!5,]{#1}\end{mdframed}}

\newcommand {\R}{\mathbb{R}}

\newcommand\tab[1][0.5cm]{\hspace*{#1}} %Tab

\author{
  Giacomo Fantoni \\
  \small Telegram: \href{https://t.me/GiacomoFantoni}{@GiacomoFantoni} \\[3pt]
  Github: \href{https://github.com/giacThePhantom/AlgoritmiStruttureDati}{https://github.com/giacThePhantom/AlgoritmiStruttureDati}}
